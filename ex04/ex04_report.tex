\documentclass[11pt,a4paper]{article}

% These are extra packages that you might need for writing the equations:
\usepackage{amsmath}
\usepackage{amsfonts}
\usepackage{amssymb}
\usepackage{booktabs}
\usepackage{hyperref}
\usepackage{listings}
\usepackage{xcolor}
\lstset {language=C++,
		 basicstyle=\ttfamily,
         keywordstyle=\color{blue}\ttfamily,
         stringstyle=\color{red}\ttfamily,
         commentstyle=\color{purple}\ttfamily,
         morecomment=[l][\color{magenta}]{\#},
       	 basicstyle=\tiny}

% You need the following package in order to include figures in your report:
\usepackage{graphicx}

% With this package you can set the size of the margins manually:
\usepackage[left=2cm,right=2cm,top=2cm,bottom=2cm]{geometry}


\begin{document}

% Enter the exercise number, your name and date here:
\noindent\parbox{\linewidth}{
 \parbox{.25\linewidth}{ \large ICP, Exercise 03 }\hfill
 \parbox{.5\linewidth}{\begin{center} \large Beat Hubmann \end{center}}\hfill
 \parbox{.2\linewidth}{\begin{flushright} \large Oct 15, 2018 \end{flushright}}
}
\noindent\rule{\linewidth}{2pt}


\section{Introduction}

The Hoshen-Kopelman algorithm~\cite{hoshen} as described in~\cite{herrmann} was implemented for a square site lattice and a series of experiments was run
to establish cluster size distributions $n_s$ as a function of different site occupation probabilities $p$.

\section{Algorithm Description}
The algorithm was implemented as described in the course lecture notes~\cite{herrmann}~(page 30) with one difference:\\
To discover the original cluster of a candidate cluster with negative mass $M_k$, a recursive function instead of a while loop was used.

\section{Results}


The program was implemented as described above and submitted with this report. 
A square site lattice of side length $L=1000$ and thus size $L^2 = 10^6$ and occupation probabilities $p \in \{0.1, 0.2, 0.3, 0.4, p_c=0.592746, 0.6, 0.7, 0.8\}$ were used.
For each occupation probability, the experiment was run $c=100$ times with differing seeds for \texttt{C++}'s Mersenne Twister \texttt{mt19937} to initialize the lattice.
The sizes of the obtained clusters were recorded and the total counts per size $\widetilde{n_s}$ over all $c$ experiments then normalized~(equation~\ref{eqn:1}) to obtain the cluster size distributions $n_s$.


\begin{equation}
n_s \leftarrow \frac{\widetilde{n_s}}{L^2 \cdot c}
\label{eqn:1}
\end{equation}

The normalized $n_s$ were plotted against cluster size $s$ (figures~\ref{fig:a},~\ref{fig:b} and \ref{fig:c}) in emulation of the figures shown in the lecture notes~\cite{herrmann}~(page 31).

\begin{figure}[ht]
\begin{center}
\includegraphics[scale=0.4]{results_a.png} 
\end{center}
\caption{Cluster size distribution for $p < p_c$.}
\label{fig:a}
\end{figure}

\begin{figure}[ht]
\begin{center}
\includegraphics[scale=0.4]{results_b.png} 
\end{center}
\caption{Cluster size distribution for $p = p_c$.}
\label{fig:b}
\end{figure}

\begin{figure}[ht]
\begin{center}
\includegraphics[scale=0.4]{results_c.png} 
\end{center}
\caption{Cluster size distribution for $p > p_c$.}
\label{fig:c}
\end{figure}

\section{Discussion}
The results were in line with the theoretical expectations from class.
For some reason unknown to me, my personal machine's OS system wouldn't let me allocate memory for $L > 1023$ and terminated with a segmentation fault 11 whenever trying to do so. As the results already were satisfactory with $L=1000$, I didn't invest any further time in trouble shooting.
\begin{thebibliography}{99}

\bibitem{herrmann}
	Herrmann, H. J.,
	Singer, H. M.,
	Mueller L.,
	Buchmann, M.-A.,\\
	\emph{Introduction to Computational Physics - Lecture Notes},\\
	ETH Zurich,\\
	2017.
	
\bibitem{hoshen}
	Hoshen, J.,
	Kopelman, R.,\\
	\emph{Percolation and cluster distribution. I. Cluster multiple labeling technique and critical concentration algorithm},\\
	Phys. Rev. B 14, 3428,\\
	1976.

\end{thebibliography}



\end{document}