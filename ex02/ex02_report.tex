\documentclass[11pt,a4paper]{article}

% These are extra packages that you might need for writing the equations:
\usepackage{amsmath}
\usepackage{amsfonts}
\usepackage{amssymb}
\usepackage{booktabs}
\usepackage{hyperref}
\usepackage{listings}
\usepackage{xcolor}
\lstset {language=C++,
		 basicstyle=\ttfamily,
         keywordstyle=\color{blue}\ttfamily,
         stringstyle=\color{red}\ttfamily,
         commentstyle=\color{purple}\ttfamily,
         morecomment=[l][\color{magenta}]{\#},
       	 basicstyle=\tiny}

% You need the following package in order to include figures in your report:
\usepackage{graphicx}

% With this package you can set the size of the margins manually:
\usepackage[left=2cm,right=2cm,top=2cm,bottom=2cm]{geometry}


\begin{document}

% Enter the exercise number, your name and date here:
\noindent\parbox{\linewidth}{
 \parbox{.25\linewidth}{ \large ICP, Exercise 02 }\hfill
 \parbox{.5\linewidth}{\begin{center} \large Beat Hubmann \end{center}}\hfill
 \parbox{.2\linewidth}{\begin{flushright} \large Oct 8, 2018 \end{flushright}}
}
\noindent\rule{\linewidth}{2pt}


\section{Introduction}

The burning method algorithm to study percolation was implemented. A series experiments on square site lattices of different sizes was conducted using the algorithm.

\section{Algorithm Description}
The algorithm was implemented as described in the course lecture notes~\cite{herrmann}~(page 25).

\section{Results}

\subsection{Task 1}

The program was implemented as instructed. A sample square site lattice with size $100 \times 100$ and occupation probability $p = 0.5$ is shown in figure~\ref{fig1}.


\begin{figure}[ht]
\begin{center}
\includegraphics[scale=0.2]{task1.eps} 
\end{center}
\caption{Sample square site lattice with side length $N = 100$ and occupation probability $p=0.5$.}
\label{fig1}
\end{figure}

\begin{figure}[ht]
\begin{center}
\includegraphics[scale=0.2]{task2_002.eps} 
\end{center}
\caption{Sample square site lattice with side length $N = 100$ and occupation probability $p=0.6$: Forest fire simulation at time step $t=2$.}
\label{fig2_1}
\end{figure}



\subsection{Task 2}
Based on the code from task 1, the program was implemented without issues. Snapshots from different time steps are shown in figure~\ref{fig2_1} and in figure~\ref{fig2_2}.


\begin{figure}[ht]
\begin{center}
\includegraphics[scale=0.2]{task2_100.eps} 
\end{center}
\caption{Sample square site lattice with side length $N = 100$ and occupation probability $p=0.6$: Forest fire simulation at time step $t=100$.}
\label{fig2_2}
\end{figure}



\subsection{Task 3}
The code was implemented and 1000 measurements were taken for each occupation probability $p \in \{0, 0.05, 0.1, 0.15, 0.2, 0.25, 0.3, 0.35, 0.4, 0.45, 0.5, 0.55, 0.6, 0.65, 0.7, 0.75, 0.8, 0.85, 0.9, 0.95, 1\}$ for different site lattice side lengths $N \in \{10, 25, 50, 100, 250, 500, 1000\}$. To populate the lattices, the Mersenne Twister pseudorandom number generator from the \texttt{C++} standard library was used. The seeds were different for each of the 1000 measurements in each series. The figures show the obtained results for each series' order~parameter~\ref{fig3_1}, average fire duration~\ref{fig3_2} and average shortest path length~\ref{fig3_3}.


\begin{figure}[ht]
\begin{center}
\includegraphics[scale=0.9]{figure3_1.eps} 
\end{center}
\caption{Measured order parameters for different site grid side lengths $N$ and initial site occupation probability $p$.}
\label{fig3_1}
\end{figure}

\begin{figure}[ht]
\begin{center}
\includegraphics[scale=0.9]{figure3_2.eps} 
\end{center}
\caption{Measured average fire duration for different site grid side lengths $N$ and initial site occupation probability $p$.}
\label{fig3_2}
\end{figure}

\begin{figure}[ht]
\begin{center}
\includegraphics[scale=0.9]{figure3_3.eps} 
\end{center}
\caption{Measured shortest path length for different site grid side lengths $N$ and initial site occupation probability $p$.}
\label{fig3_3}
\end{figure}


\section{Discussion}
The results were in line with the theoretical expectations from class. The algorithm as such already scales badly in $N$ by design; writing (sequential) \texttt{.ppm} output files makes things a lot worse. Drawing the lattices for output was therefore skipped in task 3. Additionally, I designed my main function to use a single loop for which parallelisation with \texttt{OpenMP} helped increasingly with growing $N$.

\begin{thebibliography}{99}

\bibitem{herrmann}
	Herrmann, H. J.,\\
	Singer, H. M.,\\
	Mueller L.,\\
	Buchmann, M.-A.,\\
	\emph{Introduction to Computational Physics - Lecture Notes},\\
	ETH Zurich,\\
	2017.
\end{thebibliography}

\end{document}